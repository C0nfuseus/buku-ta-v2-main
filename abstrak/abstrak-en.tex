\begin{center}
  \large\textbf{ABSTRACT}
\end{center}

\addcontentsline{toc}{chapter}{ABSTRACT}

\vspace{2ex}

\begingroup
  % Menghilangkan padding
  \setlength{\tabcolsep}{0pt}

  \noindent
  \begin{tabularx}{\textwidth}{l >{\centering}m{3em} X}
    % Ubah kalimat berikut dengan nama mahasiswa
    \emph{Name}     &:& Dimas Nazli Bahaduri \\

    % Ubah kalimat berikut dengan judul tugas akhir dalam Bahasa Inggris
    \emph{Title}    &:& \emph{Ether Wallet using Ethereum based on Proof of Stake for Validating Transaction} \\

    % Ubah kalimat-kalimat berikut dengan nama-nama dosen pembimbing
    \emph{Advisors} &:& 1. Mochamad Hariadi, ST., M.Sc., Ph.D. \\
  &:& 2. Dr. Supeno Mardi Susiki Nugroho, ST., MT.
  \end{tabularx}
\endgroup

% Ubah paragraf berikut dengan abstrak dari tugas akhir dalam Bahasa Inggris
\emph{In this research, we propose a study that uses Ethereum as a banking solution, namely a digital wallet. The banking department has developed by adopting a digital wallet which is considered good for consumers because it minimizes recorded transactions. As a solution, Ethereum is used as the main transaction system. This Ethereum adopts a distributed information system which requires that every audit information be shared with all connected users. As a result, with an information system like this, all transactions and information access can be seen to all nodes which makes it difficult to change. Later, by adopting Ethereum into a new digital wallet, it is hoped that it will eliminate problems that may be encountered from conventional wallets and provide new options for consumer users in Indonesia to transact without limits. This research will run a transaction process that starts with a request to make a transaction. When the transaction process starts, verification of transaction data from the recipient and delivery will be carried out. After the transaction is verified, continue the validation process from the nodes joined to the Ethereum network. This validation will use the Proof of Stake convention/agreement which analyzes whether the sender actually has the number to send. Then if the validation has reached the threshold, a nominal amount is listed in the transaction that will be sent from the sender to the recipient. Then the transaction process is complete. From the temporary results obtained, the implementation of some digital wallet features, namely detection of deposits to Ether and detection of digital wallet owners. From the temporary results the transaction can be completed within 13,212 seconds.}

% Ubah kata-kata berikut dengan kata kunci dari tugas akhir dalam Bahasa Inggris
\emph{Keywords}:\emph{Digital Wallet} ,\emph{Ethereum} ,\emph{Proof of Stake}  ,\emph{Blockchain}.
