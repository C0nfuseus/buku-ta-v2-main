\chapter{HASIL DAN PEMBAHASAN}
\label{chap:hasilpembahasan}

Pada bab sebelumnya telah dibuat penjelasan mengenai sistem penelitian secara keseluruhan. Mulai dari bagian antarmuka pengguna, hingga bagian smart contract telah dibuat dan didokumentasikan. Setelah itu pada bagian ini akan dijelaskan pengujian yang akan dilakukan pada sistem. Pengujian bertujuan untuk menguji apakah seluruh fitur yang akan dibuat berjalan sesuai dengan tujuan penelitian ini.

\section{Skenario Pengujian}
\label{sec:skenariopengujian}

Pada bagian pengujian, akan dilakukan pengujian sistem secara menyeluruh. Pengujian ini mencakup bagian antarmua pengguna dan fitur Smart Contract yang telah di-deploy.Pengujian ini nantinya akan dilakukan di beberapa perangkat yang telah disebutkan pada bab sebelumnya dengan variabel yang berbeda. \\
Pengujian bagian pertama adalah pengujian antarmuka pengguna apakah sudah memenuhi kebutuhan sesuai prosedur yang ditetapkan. Tujuannya agar fitur yang dibuat sudah sesuai dengan penjelasan dan use case sesuai pada bab sebelumnya.
Pengujian kedua adalah pengujian fitur Smart Contract yang merupakan pengujian berbagai fungsi yang telah ditetapkan di awal penelitian. Apakah dapat berjalan sesuai dengan ekspektasi penelitian ini.\\
Berikut adalah pengujian yang telah dilakukan.

\subsection{Pengujian Antarmuka Pengguna}
\label{subsec:ujiuiux}

Pada bagian ini pengujian antarmuka pengguna dilakukan dengan menggunakan metode Black Box testing. Black Box testing dipilih pada pengujian karena Black Box testing merupakan metode pengujian suatu aplikasi dari sisi penggunanya sendiri. Artinya pada pengujian ini, diharapkan pengujian dapat menempatkan posisinya sebagai pengguna aplikasi ini. Pada

\subsection{Pengujian Pada Jaringan Tes Kintsugi}
\label{subsec:teskintsugi}

Berikut adalah hasil pengujian pada jaringan tes PoS Kintsugi.

\begin{longtable}{|c|c|c|}
  \caption{Hasil Pengujian Pada Jaringan Kintsugi}
  \label{tb:kintsugitest}\\
  \hline
  \rowcolor[HTML]{C0C0C0}
  \textbf{Nomor} & \textbf{User 1} & \textbf{User 2} \\
  \hline
  1 & Sukses & Sukses \\
  2 & Sukses & Sukses \\
  3 & Sukses & Sukses \\
  \hline
\end{longtable}

Sembari melakukan transaksi dilakukan pengukuran durasi setiap transaksi yang telah sukses

\begin{longtable}{|c|c|}
  \caption{Hasil Pengukuran Durasi Transaksi di Jaringan Tes Kintsugi}
  \label{tb:kintsugispeed}\\
  \hline
  \rowcolor[HTML]{C0C0C0}
  \textbf{Nomor} & \textbf{Kecepatan} \\
  \hline
  1 & 20,27 detik  \\
  2 & 26,57 detik\\
  3 & 35,07 detik \\
  \hline
\end{longtable}

\subsection{Pengujian Pada Jaringan Tes Kiln}
\label{subsec:teskiln}

Berikut adalah hasil pengujian pada jaringan tes PoS Kintsugi.

\begin{longtable}{|c|c|c|}
  \caption{Hasil Pengujian Deployment Smart Contract Pada Jaringan Kiln}
  \label{tb:kilntest}\\
  \hline
  \rowcolor[HTML]{C0C0C0}
  \textbf{Nomor} & \textbf{Status}\\
  \hline
  1 & Sukses \\
  2 & Sukses \\
  3 & Sukses \\
  4 & Sukses \\
  5 & Sukses \\
  6 & Gagal \\
  7 & Sukses \\
  \hline
\end{longtable}

Setelah transaksi dapat terjadi dilakukan pengukurang durasi 

\begin{longtable}{|c|c|}
  \caption{Hasil Pengukuran Durasi Deployment Smart Contract di Jaringan Tes Kiln}
  \label{tb:kilnspeed}\\
  \hline
  \rowcolor[HTML]{C0C0C0}
  \textbf{Nomor} & \textbf{Kecepatan} \\
  \hline
  1 & 13,33 detik \\
  2 & 13,33 detik \\
  3 & 13,33 detik \\
  4 & 13,33 detik \\
  5 & 13,33 detik \\
  6 & 13,33 detik \\
  7 & 13,33 detik \\
  \hline
\end{longtable}