\chapter{PENDAHULUAN}
\label{chap:pendahuluan}

% Ubah bagian-bagian berikut dengan isi dari pendahuluan

\section{Latar Belakang}
\label{sec:latarbelakang}

Di era industri 4.0 dan perkembangan 5G di Indonesia, saat ini Indonesia sedang bersiap menjadi sebuah negara yang berkembang untuk mengadopsi teknologi untuk menunjang sektor industri dan sektor bisnis. Sektor bisnis dan sektor industri di Indonesia sedang mengalami pemulihan yang ditandai dengan beberapa program nasional pemulihan pasca COVID-19. Program pemulihan ekonomi pasca COVID-19 ini bertujuan untuk mempersiapkan Indonesia bersaing di kancah dunia. Dalam proses perkembangan Indonesia untuk bersaing di panggung dunia, tentu diiringi dengan perkembangan teknologi perbankan yaitu pembuatan dompet digital. \\
Serupa dengan dompet pada umumnya yang menyimpan uang, dompet digital ini adalah aplikasi untuk menyimpan uang, melakukan transaksi, kirim uang, dan sebagainya. Dompet digital ini banyak digunakan untuk kegiatan tersebut dan melansir dari Kontan.co.id, kegiatan dan transaksi di salah satu dompet digital mengalami kenaikan hingga 91\% dan MAU sebesar 73\%. Ini bisa jadi menandakan pertumbuhan ekonomi dan bisnis di Indonesia. Dompet digital ini juga menyediakan fitur masing-masing. Namun dari segelintir fitur ini dompet digital yang ada di Indonesia belum bisa menjangkau transfer ke luar negeri. Padahal beberapa produk Indonesia sudah mulai dilirik di beberapa negara tetangga. \\
Seperti yang diketahui bahwa Indonesia sedang bersiap menuju pemulihan pasca COVID-19 dan diperlukan fasilitas penunjang kegiatan ekonomi berbasis teknologi. Aplikasi dari teknologi yang telah ada untuk menjamin kenyamanan dan keamanan kegiatan bisnis. Teknologi harus mampu mengatasi beberapa aspek dasar dalam kegiatan transaksi mancanegara.\\
Terdapat teknologi untuk mendukung kebutuhan ini yaitu \emph{Blockchain}. Teknologi ini berdasarkan transfer file peer-to-peer yang aman. Dan saat ini sudah banyak jenis \emph{Blockchain} yang digunakan di dunia mulai dari \emph{Bitcoin, Ethereum, Solana, hingga Pancakeswap}. Semua \emph{Blockchain} ini punya implementasi mulai dari pengiriman pesan sederhana, \emph{public ledger}, hingga dompet digital. 

\section{Permasalahan}
\label{sec:permasalahan}

Berdasarkan pendahuluan diatas,dapat ditarik sebuah permasalahan yaitu perlu adanya sebuah alat transaksi yang mengadopsi sistem informasi yang terdistribusi atau \emph{Blockchain}. Alat transaksi tersebut adalah dompet digital yang berbasis \emph{Ethereum} menggunakan konsensus/kesepakatan \emph{Proof of Stake} sebagai cara untuk validiasi transaksinya.

\section{Batasan Masalah}
\label{sec:batasanmasalah}

Untuk memfokuskan permasalahan yang diangkat maka dilakukan pembatasan masalah. Berikut batasan-batasan masalah tersebut diantaranya adalah:

\begin{enumerate}[nolistsep]

  \item \emph{Blockchain} beserta \emph{Cryptocurrencies} yang digunakan dalam penelitian ini adalah \emph{Ethereum} dan \emph{Ether} beserta turunan dari Ether tersebut.
  \item Jenis jaringan yang digunakan adalah jaringan \emph{testnet} yang tersedia
  \item Sistem dompet hanya akan melakukan fitur berupa mengirim uang, mendeteksi akun, dan mengirimkan sejumlah \emph{Ether}
  \item Aplikasi dompet digital akan berbentuk website.
  \item Aplikasi penghubung antara \emph{Ethereum} dan aplikasi dompet digital yaitu \emph{Metamask}.

\end{enumerate}

\section{Tujuan}
\label{sec:Tujuan}

Dari penelitian yang akan dilakukan kali ini, penulis ingin mencapai tujuan yaitu membuktikan penggunaan \emph{Blockchain Ethereum} sebagai sistem alat transaksi dengan mengirimkan sejumlah \emph{Ether} antar pengguna dan mengetahui nilai optimal gas yang digunakan untuk transaksi sehari-hari.

\section{Manfaat}
\label{sec:manfaat}

Manfaat dari penelitian ini adalah membuat sebuah sistem dompet digital yang menggunakan \emph{Ethereum} yang bisa digunakan untuk transaksi digital di Indonesia.

% Format Buku TA baru, ga pake sistematika penulisan

% \section{Sistematika Penulisan}
% \label{sec:sistematikapenulisan}

% Laporan penelitian tugas akhir ini terbagi menjadi \lipsum[1][1-3] yaitu:

% \begin{enumerate}[nolistsep]

%   \item \textbf{BAB I Pendahuluan}

%   Bab ini berisi \lipsum[2][1-5]

%   \vspace{2ex}

%   \item \textbf{BAB II Tinjauan Pustaka}

%   Bab ini berisi \lipsum[3][1-5]

%   \vspace{2ex}

%   \item \textbf{BAB III Desain dan Implementasi Sistem}

%   Bab ini berisi \lipsum[4][1-5]

%   \vspace{2ex}

%   \item \textbf{BAB IV Pengujian dan Analisa}

%   Bab ini berisi \lipsum[5][1-5]

%   \vspace{2ex}

%   \item \textbf{BAB V Penutup}

%   Bab ini berisi \lipsum[6][1-5]

% \end{enumerate}
