\begin{center}
  \Large
  \textbf{KATA PENGANTAR}
\end{center}

\addcontentsline{toc}{chapter}{KATA PENGANTAR}

\vspace{2ex}

% Ubah paragraf-paragraf berikut dengan isi dari kata pengantar

Puji dan syukur kehadirat penyusun sampaikan kepada Allah SWT. karena atas limpahan berkat, rahmat, dan hidayah-Nya penulis dapat menyelesaikan penelitian berjudul \textbf{ \emph{Ether Wallet} Menggunakan \emph{Ethereum} Berbasis \emph{Proof of Stake} untuk Validasi Transaksi} ini tepat waktu.

Penelitian ini disusun dalam rangka pemenuhan bidang riset di Departemen Teknik Komputer, serta digunakan sebagai persyaratan menyelesaikan pendidikan S1. Penelitian ini dapat terselesaikan tidak lepas dari bantuan berbagai pihak. Oleh karena itu, penulis mengucapkan terima kasih kepada:

\begin{enumerate}[nolistsep]

  \item Keluarga, Ibu, Abang dan Kakak tercinta yang telah memberikan dukungan moril maupun dukungan material dalam menyelesaikan penelitian ini. 

  \item Bapak Supeno Mardi Susiki Nugroho, S.T., M.T. selaku Kepala Departemen Teknik Komputer, Fakultas Teknologi Elektro dan Informatika Cerdas (FTEIC), Institut Teknologi Sepuluh Nopember.

  \item Bapak Mochamad Hariadi, S.T., M.Sc., Ph.D. selaku dosen pembimbing I dan Bapak Dr. Supeno Mardi Susiki Nugroho, S.T., M.T. selaku  dosen pembimbing II yang memberikan arahan dan petuah selama mengerjakan penelitian tugas akhir ini. 

 \item Bapak dan ibu dosen pengajar Departemen Teknik Komputer, atas pengajaran, bimbingan, serta perhatian yang diberikan selama ini.

 \item Secara khusus penulis mengucapkan terima kasih kepada teman teman satu bimbingan terutama Rizqullah dan Ananta, yang telah berbagi ilmu dan pikirannya selama ini. Tiada dirimu, diriku hanya diam termenung.

 \item Teman-teman dari seluruh kalangan yang tidak bisa penulis sebutkan satu per satu namun tidak membuat penulis lupa dukungan yang terus kalian suguhkan baik melalui kesenangan permainan maupun curhatan melantur di tengah malam.	 

\end{enumerate}

Sebagaimana pepatah berkata "Tiada Gading yang Tak Retak", maka dari itu penulis mengharapkan kritik dan saran yang membangun. Akhir kata, semoga penelitian ini dapat digunakan semestinya dan bermanfaat bagi semua. Amin. 

\begin{flushright}
  \begin{tabular}[b]{c}
    % Ubah kalimat berikut dengan tempat, bulan, dan tahun penulisan
    Surabaya, Mei 2022\\ 
    \\
    \\
    \\
    \\
    % Ubah kalimat berikut dengan nama mahasiswa
    Dimas Nazli Bahaduri
  \end{tabular}
\end{flushright}
